\PassOptionsToPackage{unicode=true}{hyperref} % options for packages loaded elsewhere
\PassOptionsToPackage{hyphens}{url}


\documentclass[ignorenonframetext,]{beamer}
\setbeamertemplate{caption}[numbered]
\setbeamertemplate{caption label separator}{: }
\setbeamertemplate{itemize items}[square]
\setbeamercolor{caption name}{fg=normal text.fg}
\beamertemplatenavigationsymbolsempty

\usepackage{bibentry}

\usepackage{lmodern}
\usepackage{stmaryrd}
\usepackage{amssymb,amsmath}
\usepackage{ifxetex,ifluatex}
\usepackage{fixltx2e} % provides \textsubscript
\ifnum 0\ifxetex 1\fi\ifluatex 1\fi=0 % if pdftex
  \usepackage[T1]{fontenc}
  \usepackage[utf8]{inputenc}
  \usepackage{textcomp} % provides euro and other symbols
\else % if luatex or xelatex
  \usepackage{unicode-math}
  \defaultfontfeatures{Ligatures=TeX,Scale=MatchLowercase}
\fi
\usetheme[]{CambridgeUS}
% use upquote if available, for straight quotes in verbatim environments
\IfFileExists{upquote.sty}{\usepackage{upquote}}{}
% use microtype if available
\IfFileExists{microtype.sty}{%
\usepackage[]{microtype}
\UseMicrotypeSet[protrusion]{basicmath} % disable protrusion for tt fonts
}{}
\IfFileExists{parskip.sty}{%
\usepackage{parskip}
}{% else
\setlength{\parindent}{0pt}
\setlength{\parskip}{6pt plus 2pt minus 1pt}
}
\usepackage{hyperref}
\hypersetup{
            pdftitle={Refine your types!},
            pdfauthor={Romain Ruetschi},
            pdfborder={0 0 0},
            breaklinks=true}
\urlstyle{same}  % don't use monospace font for urls
\newif\ifbibliography
\usepackage{color}
\usepackage{fancyvrb}
\newcommand{\VerbBar}{|}
\newcommand{\VERB}{\Verb[commandchars=\\\{\}]}
\DefineVerbatimEnvironment{Highlighting}{Verbatim}{commandchars=\\\{\}}
\widowpenalties 1 10000
\raggedbottom
\setbeamertemplate{part page}{
\centering
\begin{beamercolorbox}[sep=16pt,center]{part title}
  \usebeamerfont{part title}\insertpart\par
\end{beamercolorbox}
}
\setbeamertemplate{section page}{
\centering
\begin{beamercolorbox}[sep=12pt,center]{part title}
  \usebeamerfont{section title}\insertsection\par
\end{beamercolorbox}
}
\setbeamertemplate{subsection page}{
\centering
\begin{beamercolorbox}[sep=8pt,center]{part title}
  \usebeamerfont{subsection title}\insertsubsection\par
\end{beamercolorbox}
}
\useinnertheme{rectangles}
\useoutertheme{infolines}
\definecolor{darkred}{rgb}{0.8,0,0}
\usecolortheme[named=darkred]{structure}

\usepackage{lmodern}
\usepackage{amssymb,amsmath}
\usepackage{listings,multicol}
\usepackage{lstscala}

\lstdefinelanguage{PureScala}{ % Using `Scala` result in a infinite recursion
  style=scala-color,
  morekeywords={[2]Unit,Boolean,Byte,Int,BigInt,String,Char,Option,List,true,false}
}

\newcommand{\scala}[1]{\text{\lstinline[language=PureScala]|#1|}}

% Env for short code: no label, no caption and "inlined" with a box
\lstnewenvironment{Code}[1]
    {
      \lstset{
        language=#1,
        basicstyle=\ttfamily,
        frame=trBL % box the frame & double line on bottom and left side
      }
    }
    {
    }


\title{Refine your types!}
\author{Romain Ruetschi}
\date{31.05.2018}

\AtBeginSection[]{
  \begin{frame}
  \vfill
  \centering
  \begin{beamercolorbox}[sep=8pt,center]{title}
    \usebeamerfont{title}\secname\par%
  \end{beamercolorbox}
  \vfill
  \end{frame}
}

\begin{document}

\frame{\titlepage}

\frame{
  \frametitle{Two words about me}

  Hi, my name is Romain Ruetschi, but you can also call me Romac.\\
  \vspace{10pt}
  I am usually found online under various spellings of ``romac''.\\
  \vspace{20pt}
  \begin{itemize}
  \setlength{\itemindent}{5em}
  \item[Twitter:]  \url{https://twitter.com/_romac}
  \item[GitHub:]   \url{https://github.com/romac}
  \item[Homepage:] \url{https://romac.me}
  \end{itemize}
}

\begin{frame}
  \vfill
  \centering
  \begin{beamercolorbox}[sep=8pt,center]{title}
    \usebeamerfont{title}Refinement types\par%
  \end{beamercolorbox}
  \vfill
\end{frame}

\section[Outline]{}
\frame{
  \frametitle{Refinement types} % TODO: Remove outline from TOC
  \tableofcontents
}

\section{A quick introduction}

\frame
{
  \frametitle{Types}
  
  $$\scala{Int}$$ \pause
  $$\scala{Boolean}$$ \pause
  $$\scala{List[Int]}$$ \pause
  $$\scala{List[A]} \rightarrow \scala{Int}$$
}

\frame
{
  \frametitle{Types}
  
  $$3: \scala{Int}$$ \pause
  $$\scala{true}: \scala{Boolean}$$ \pause
  $$\scala{List}(1,2,3): \scala{List[Int]}$$ \pause
  $$\scala{(xs: List[A]) => xs.length}: \scala{List[A]} \rightarrow \scala{Int}$$
}

\frame
{
  \frametitle{Refinement types}
  
  $$\{\ x: T\ |\ p\ \}$$
}

\frame
{
  \frametitle{Refinement types}
  
  $$\{\ n: \scala{Int}\ |\ n > 0  \ \}$$ \pause
  $$\{\ xs: \scala{List[A]}\ |\ \scala{!$xs$.isEmpty} \ \}$$ \pause
  $$\{\ t: \scala{Tree}\ |\ \scala{isBalanced($t$)} \ \}$$
}

\frame
{
  \frametitle{Refinement types}
  
  $$\{\ xs: \scala{List[A]}\ |\ \scala{$xs$.length} \not= 0 \} \rightarrow A$$
  \vspace{20pt}
  \pause
  $$\scala{List[A]} \rightarrow \{\ n: \scala{Int}\ |\ n \ge 0 \}$$
  
}

\frame
{
  \frametitle{Refinement types}
  
  $$A \rightarrow \{\ n: \scala{Int}\ |\ x \ge 0 \} \rightarrow \{\ xs: \scala{List[A]}\ |\ \scala{$xs$.length} = n \ \} $$
}

\frame
{
  \frametitle{Relation with dependent types}
  
    Not easy to precisely define either system. One view is that:
    
    \begin{itemize}
        \item With dependent types, types can refer to terms, the calculus is normalizing.
        \item With refinement types, types don't necessarily need to be able to refer to terms, and the calculus does not need to be normalizing, because proofs are discharged to a \textit{solver}.
        \item In practice, it is natural to allow refinement types to refer to terms.
      \end{itemize}
}

\frame
{
  \frametitle{Relation with dependent types, continued}
  
 If we restrict ourselves to the view that \emph{dependent types} $\approx$ \emph{Coq} and \emph{refinement types} $\approx$ \emph{LiquidHaskell}, then:
      \begin{itemize}
        \item Dependent types are more expressive than refinement types, ie. one can model pretty much any kind of mathematics using dependent types, tactics, and manual proofs.
        \item Refinement types are more suited for automation, as predicates are drawn from a decidable logic, and proof obligations can thus be discharged to an SMT solver.
      \end{itemize}
}


\section{Under the hood}

\frame
{
  \frametitle{Subtyping}
  
  Refinement types rest on the following notion of \emph{subtyping}:
  
  
  $$\Gamma \vdash \{\ x: A\ |\ p\ \} \preceq \{\ y: A\ |\ q\ \}$$
  $$\Leftrightarrow$$
  $$Valid\Big(\llbracket\Gamma\rrbracket \wedge \llbracket p \rrbracket \Rightarrow \llbracket q \rrbracket\Big)$$
  $$\Leftrightarrow$$
  $$CheckSat\Big(\neg(\llbracket\Gamma\rrbracket \wedge \llbracket p \rrbracket \Rightarrow \llbracket q \rrbracket)\Big) = UNSAT$$
}

\frame
{
  \frametitle{Subtyping (example)}

  $$\{\ \scala{val x: Int = 42\ \}} \vdash \scala{\{ y: Int \| y > x \} $\ \preceq\ $ \{ z: Int \| z > 0 \}}$$
  $$\Leftrightarrow$$
  $$Valid\big((x = 42\ \wedge\  y > x\ \wedge\ z = y)\ \Rightarrow\ z > 0)$$
  $$\Leftrightarrow$$
  $$CheckSat\Big(\neg((x = 42\ \wedge\  y > x\ \wedge\ z = y)\ \Rightarrow\ z > 0)\Big) = UNSAT$$
}

\frame
{
  \frametitle{Contracts}
  
  This function
  $$\scala{def f(x: Int \{ x > 0 \}): \{ y: Int \| y < 0 \} = 0 - x}$$
  is correct if
  $$\{\ \scala{x: Int \| x > 0}\ \}\ \vdash\ \{\ \scala{z: Int \| z = 0 - x}\ \}\ \preceq\ \{\ \scala{y: Int \|\ y < 0}\ \}$$
  $$\Leftrightarrow$$
  $$Valid\Big((x > 0\ \wedge\ (z = 0 - x)\ \wedge\ (y = z))\ \Rightarrow\ y < 0 \Big)$$
}

\frame{
  \frametitle{Solving constraints with SMT solvers}
  
  \begin{itemize}
    \item Satisfiability Modulo Theories: Akin to a SAT solver with support for additional theories: algebraic data types, integer arithmetic, real arithmetic, bitvectors, sets, etc.
    \item Can choose from Z3, CVC4, Yices, Princess, and others.
    \item In practice, cannot just translate from the host language into SMT because of quantifiers, recursive functions, polymorphism, etc.
    \item Lots of work, difficult to get right (ie. sound and complete).
  \end{itemize}
}

\frame
{
  \frametitle{Solving constraints with Inox\footnote{\url{https://github.com/epfl-lara/inox}}}
  
  \begin{enumerate}
    \item Solver for higher-order functional programs which provides first-class support for features such as:
    
    \begin{enumerate}
    \item Recursive and first-class functions
    \item ADTs, integers, bitvectors, strings, set-multiset-map abstractions
    \item Quantifiers
     \item ADT invariants
     \end{enumerate}
    \item Implements a very involved \textit{unfolding strategy} to deal with all of the above \cite{leon, inox1, inox2}
     \item Interfaces with various SMT solvers (Z3, CVC4, Princess)
     \item Powers Stainless, a verification system for Scala\footnote{\url{https://github.com/epfl-lara/stainless}}
  \end{enumerate}
}

\frame
{
  \frametitle{Write your own language with refinement types}
  
  Demo
}

\section{In the wild}

\frame
{
  \frametitle{LiquidHaskell}
  
  \begin{itemize}
    \item Modern incarnation of refinement types for Haskell, ie. \textit{Liquid types} \cite{liquid}
    \item Refinement are quantifier-free predicates drawn from a decidable logic. \cite{liquid}
    \item Type refinement are specified as comments in the source code.
  \end{itemize}
}

\frame
{
  \frametitle{LiquidHaskell}
  
  Demo
}

\frame
{
  \frametitle{$\text{F}^\ast$}
  
  \begin{itemize}
    \item General-purpose dialect of ML with effects aimed at program verification.
    \item Dependently-typed language with refinements, type checking done via an SMT solver.
    \item Can be extracted to efficient OCaml, F#, or C code.
    \item Initially developed at Microsoft Research.
  \end{itemize}
}

\frame
{
  \frametitle{$\text{F}^\ast$}
  
  \begin{itemize}
  \item Part of \textit{Project Everest}, an in-progress verified implementation of HTTPS, TLS, X.509, and cryptographic algorithms.\footnote{\url{https://project-everest.github.io}}
  \end{itemize}
}

\frame
{
  \frametitle{Scala 3 (one day?)}
    \begin{itemize}
    \item Ongoing effort by Georg Schmid to add refinement types to Dotty/Scala 3\cite{georg}
    \end{itemize}
  
}

\frame
{
  \frametitle{Acknowledgement}
  
  Many thanks to Georg Schmid for his insights and for taking time to answer my questions.\\
  \vspace{10pt}
  Go check out his work! \cite{georg}
}

\begin{frame}
  \vfill
  \centering
  \begin{beamercolorbox}[sep=8pt,center]{title}
    \usebeamerfont{title}Thanks!\par%
  \end{beamercolorbox}
  \vfill
\end{frame}

\begin{frame}[allowframebreaks]
  \frametitle{References}
  \setbeamertemplate{bibliography item}[text]
  \bibliographystyle{ieeetr}
  \bibliography{bibliography.bib}
\end{frame}

\end{document}
